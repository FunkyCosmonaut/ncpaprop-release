\section{Installation notes}

\subsection{Prerequisites}

\subsubsection{Operating System}

\textbf{ncpaprop} is designed to run in a UNIX environment. It has been tested on Ubuntu, Red Hat, and CentOS Linux operating systems, and MacOS Mojave (10.14). 

\subsubsection{Libraries}

Fourier transforms are performed in \textbf{ncpaprop} using the \textbf{FFTW} libraries, http://www.fftw.org. The \textbf{GSL} libraries, http://www.gnu.org/software/gsl/, are used for interpolation and splining. The \textbf{SLEPc} library, http://www.slepc.upv.es/, is used to solve the large dimensional eigenvalue problems required for the normal mode models. \textbf{SLEPc} requires the \textbf{PETSc} libraries, http://www.mcs.anl.gov/petsc/, to which it can be seen as a submodule. \textbf{PETSc} routines are also used to perform the matrix calculations required by the PE model.

\subsubsection{Compilers}

\textbf{ncpaprop} is written in C++ and was built and tested using the GNU C++ compiler \textbf{g++}, and will also compile under the MacOS native \textbf{clang} compiler.  The \textbf{PETSc} and \textbf{SLEPc} libraries require a corresponding C compiler.

\subsection{Procedure}

\textbf{ncpaprop} uses a standard GNU-style configure script to detect and locate the above prerequisites and generate appropriate makefiles.  The \texttt{configure} script will search the standard library locations. An additional flag can be invoked (for Linux installations only) to allow the installer to automatically download and install the prerequisites using the native package manager. The details of this process are explained below.

\subsubsection{Installation Overview}

To begin installation, select an installation location for the \textbf{ncpaprop} directory to reside.  Clone the GitHub repository's master branch with

\texttt{git clone https://github.com/chetzer-ncpa/ncpaprop.git ncpaprop}

\noindent
and cd into \texttt{ncpaprop}.  Configuration will depend on whether the user has a properly-compiled pre-existing PETSc and SLEPc installation or not.  Properly-compiled, in this context, means compiled with C++ support and for both real and complex data types.  If this is the case, and the user wishes to link to those installations, the following environmental variables should be set:
\begin{description}
\item[\texttt{PETSC\_DIR}]\hfill \\
\noindent
The root directory of the \textbf{PETSc} installation.

\item[\texttt{SLEPC\_DIR}]\hfill \\
\noindent
The root directory of the \textbf{SLEPc} installation.

\item[\texttt{PETSC\_ARCH\_REAL}]\hfill \\
\noindent
The architecture name selected for the (default) \texttt{--with-scalar-type=real} \textbf{PETSc} build.  This frequently defaults to \texttt{arch-\$OS-c-real} or \texttt{arch-\$OS-c-debug} if the user does not specify, where \texttt{\$OS} represents the operating system, and will appear as a subdirectory of \texttt{\$PETSC\_DIR} and \texttt{\$SLEPC\_DIR}.

\item[\texttt{PETSC\_ARCH\_COMPLEX}]\hfill \\
\noindent
The architecture name selected for the \texttt{--with-scalar-type=complex} \textbf{PETSc} build.  This frequently defaults to \texttt{arch-\$OS-c-complex} or \texttt{arch-\$OS-c-complex-debug} if the user does not specify, where \texttt{\$OS} represents the operating system, and will appear as a subdirectory of \texttt{\$PETSC\_DIR} and \texttt{\$SLEPC\_DIR}.
\end{description}

The user may then simply issue the command

\texttt{./configure}

\noindent
to set up the build environment and make sure all prerequisites are met.  Alternately, the user may request that local instances of PETSc and SLEPc be installed in the \texttt{ncpaprop} directory tree with the \texttt{--with-localpetsc} flag.  Additional flags to control this process include:

\begin{description}
\item[\texttt{--enable-autodependencies}]\hfill \\
\noindent
Attempt to download and install missing prerequisites automatically. Use this option only if installation in the default installation locations is satisfactory. The \textbf{ncpaprop} installer currently supports the \textbf{apt-get} and \textbf{yum} package managers on Linux.  This option is not supported on MacOS and such users should install \textbf{FFTW} and \textbf{GSL} using their method of choice.  Use of this option on an unsupported system will result in an error.

\item[\texttt{LDFLAGS="-L/path/to/library"}]\hfill \\
\noindent
The user can also specify additional library search paths using this syntax.
\end{description}

\noindent To have the installer download and build \textbf{PETSc} and \textbf{SLEPc} local to \textbf{ncpaprop}, use approriate flags and options from the following:

\begin{description}
\item[\texttt{--with-localpetsc}]\hfill \\
\noindent
Download and build the current \texttt{maint} branch of \textbf{PETSc}.  Use of this flag also sets \texttt{--with-localslepc}.

\item[\texttt{--with-localslepc}]\hfill \\
\noindent
Download and build the latest version of \textbf{SLEPc}, which at the time of release is \textbf{3.12.2}.

\item[\texttt{--with-localpetscdebug}]\hfill \\
\noindent
Build the \texttt{debug} version of \textbf{PETSc} rather than the optimized version.

\item[\texttt{--enable-localpetscmpi}]\hfill \\
\noindent
Build \textbf{PETSc} with MPI support.  If this flag is not set, the installer will build \textbf{PETSc} with MPI support disabled..

\item[\texttt{local\_petsc\_dir=VALUE}]\hfill \\
\noindent
Override the default \textbf{PETSc} installation location (defaults to \texttt{\$NCPAPROP\_ROOT/extern/petsc}).

\item[\texttt{local\_slepc\_dir=VALUE}]\hfill \\
\noindent
Override the default \textbf{SLEPc} installation location (defaults to \texttt{\$NCPAPROP\_ROOT/extern/slepc/slepc-\$VERSION)}.

\item[\texttt{local\_petsc\_arch\_real=VALUE}]\hfill \\
\noindent
Override the default \textbf{PETSc} real architecture name (defaults to \texttt{arch-\$OS-real} or \texttt{arch-\$OS-real}).

\item[\texttt{local\_petsc\_arch\_complex=VALUE}]\hfill \\
\noindent
Override the default \textbf{PETSc} complex architecture name (defaults to \texttt{arch-\$OS-complex} or \texttt{arch-\$OS-complex}).

\item[\texttt{local\_slepc\_version=VALUE}]\hfill \\
\noindent
Override the default \textbf{SLEPc} version number for determining the download filename.  This will generally be kept up-to-date by the \textbf{ncpaprop} maintainers, but is provided in the event of maintenance lag.
\end{description}

\noindent The installation process will check the system for the presence of BLAS libraries and MPI executables; whatever is not found will be automatically downloaded and installed locally by the \textbf{PETSc} installer.  This process is independent of the \texttt{--with-autodependencies} flag, which applies only to the direct prerequisites of \textbf{ncpaprop}.  If this is undesirable, \textbf{PETSc} and \textbf{SLEPc} should be installed manually.  In general, \textbf{PETSc} and \textbf{SLEPc} are orders of magnitude more complex than \textbf{ncpaprop}, and only limited support for their installation can be provided by the \textbf{ncpaprop} developers.  Complex questions involving incompatibilies with the user's system are best directed to \textbf{PETSc} support.

\subsubsection{Compilation}

Once the configuration script has successfully run, a set of environmental variables will be presented to the user to be set in their login file.  These variables will be used by the Makefiles to indicate where to find the \textbf{PETSc}/\textbf{SLEPc} libraries, and which version (real vs complex) to use, and are necessary for future recompilations to work properly without re-invoking the \texttt{./configure} script.  The package can then be built simply by running

\texttt{make}


\subsection{Typesetting this Manual}

This manual was produced with LaTeX. The required \verb+.tex+ and graphics files are archived in the \textbf{ncpaprop} package in the \verb+manual+ directory. The main \verb+.tex+ file, \verb+NCPA_prop_manual.tex+, is contained in the subdirectory \verb+ncpaprop_run_manual_TeX+ along with a utitlity script \verb+runtex.sh+. Entering \verb+. runtex.sh+ produces the manual as a pdf file and moves it to the \verb+manual+ directory. Entering \verb+. runtex.sh bib+ produces the bibliography using bibtex. 