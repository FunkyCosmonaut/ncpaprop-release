\section{Introduction}
\label{sec: intro}

\textbf{ncpaprop} is a software package aiming at providing a comprehensive set of tested and validated numerical models for  simulating the long range propagation of infrasonic signals through the earth's atmosphere. The algorithms implemented in \textbf{ncpaprop} are designed for frequencies large enough that the effects of buoyancy can be neglected and small enough that propagation to ranges of hundreds to thousands of kilometers is possible without significant signal attenuation. Nominally,  \textbf{ncpaprop} can, without modification, be used to efficiently model narrowband propagation from 0.1 to 10 Hz and broadband propagation from 0.05 Hz to 2 or 3 Hz. The models become increasingly inefficient with increasing frequency so that run times can become prohibitive if higher frequencies are considered. 

The intent behind \textbf{ncpaprop} is to provide reliable software engines for the modeling of infrasound propagation rather than a user-friendly working environment. As such no graphical interfaces are included. Rather \textbf{ncpaprop} provides a suite of UNIX style command line programs that can be scripted into user-supplied graphical interfaces as desired. \textbf{ncpaprop} is, at this time, not stand-alone, but makes use of several publicly available numerical libraries. \textbf{ncpaprop} proper is written in C++; however, some of the programs contained in the external libraries are written in Fortran. 

Atmospheric state models are user provided and can be input as series of ascii files. Ascii file input provides the user with a high degree of flexibility, allowing the use of any atmospheric model. Stratified and range dependent atmospheric models are supported, depending on the package. Winds are treated both in the effective sound speed approximation, in which the influence of the winds are approximated by adding the horizontal component of the along-path wind to the sound speed \cite{Pierce,Godin_eff_fluid} and rigorously in what will be referred to as high Mach number models. Atmospheric attenuation can be implemented through user provided ascii files or through the built-in Sutherland-Bass model \cite{bass_suth}.

The full wave models provided by \textbf{ncpaprop} consist of a suite of normal mode models of increasing complexity and a Parabolic Equation (PE) model. All the full wave models are currently restricted to the planar (2-d) approximation in which it is assumed that propagation paths do not deviate from a fixed vertical plane. In effect, the influence of cross winds are being neglected. Effective sound speed and high Mach number normal mode models are provided in the stratified approximation. The PE model is an effective sound speed model that supports both stratified and range-dependent atmospheric specifications. Atmospheric attenuation is handled in the normal mode models using first order perturbation theory. Attenuation is treated non-perturbatively in the PE model. 

This manual is organized as follows. There is a chapter detailing the installation of the package. This is followed by a chapter discussing how the code is generally run and what atmospheric specifications are used. Then, each individual program is discussed in a chapter containing a mathematical inroduction, details on running the program, and examples. Each program provides a manual, or help, page which is reproduced and discussed. 